\documentclass[12pt]{article}
\pdfminorversion=5 
\pdfcompresslevel=9
\pdfobjcompresslevel=2
\usepackage{amssymb}
\usepackage{amsmath}
\usepackage{geometry}
\usepackage{graphicx}
\usepackage{color}
\usepackage[T1]{fontenc}
\usepackage[utf8]{inputenc}
\usepackage{mathtools}
\usepackage{lmodern}
\usepackage{caption}
\usepackage[table]{xcolor}
\usepackage{subfigure}
\usepackage{bigints}
\usepackage{bbm}
\usepackage{xr}
\usepackage{lipsum}
\usepackage{pdfpages}
\usepackage{xcolor}
\usepackage{mathrsfs}
\definecolor{redd}{rgb}{0.8, 0.1, 0.1}
\definecolor{navyblue}{rgb}{0, 0.6, 0.2}
\definecolor{amaranth}{rgb}{0.9, 0.17, 0.31}
\definecolor{alizarin}{rgb}{0.82, 0.1, 0.26}
\definecolor{bostonuniversityred}{rgb}{0.8, 0.0, 0.0}
\definecolor{brickred}{rgb}{0.8, 0.25, 0.33}
\definecolor{cornellred}{rgb}{0.7, 0.11, 0.11}
\usepackage[colorlinks,linkcolor=navyblue,urlcolor=blue,citecolor=navyblue]{hyperref}
\newcommand{\navy}[1]{\textcolor{blue}{\bf #1}}
\newcommand{\navymth}[1]{\textcolor{blue}{#1}}
\newcommand{\red}[1]{\textcolor{red}{#1}}
\usepackage{subfigure}
\usepackage[authoryear,round]{natbib}
\usepackage{sectsty}
\usepackage{multirow}
\usepackage{rotating}
\usepackage[]{morefloats}
\usepackage{booktabs}
\usepackage{float}
%\usepackage[runin]{abstract}
%\abslabeldelim{\;}
\usepackage{fancyhdr}
\usepackage{amsmath}
\usepackage{threeparttable}
%\usepackage{mathptmx}
%\usepackage{newtxmath}
%\usepackage{times}
\usepackage{tikz}
\usetikzlibrary{shapes.geometric, arrows}
\usepackage{rotating}
\UseRawInputEncoding
\usepackage{tabularx}
\usepackage{tabulary}
\usepackage[newcommands]{ragged2e}
\usepackage{tabularx}
\usepackage{adjustbox}
\usepackage{setspace} 
\doublespacing
\usepackage{mathpazo}
\usepackage{etoolbox}

\setcounter{MaxMatrixCols}{12}





\geometry{left=1.0in,right=1.0in,top=1.0in,bottom=1.0in}

\parskip5pt
\parindent15pt
\renewcommand{\baselinestretch}{1.1}

\newcommand*{\theorembreak}{\usebeamertemplate{theorem end}\framebreak\usebeamertemplate{theorem begin}}

\newcommand{\newtopic}[1]{\textcolor{Green}{\Large \bf #1}}


\definecolor{pale}{RGB}{235, 235, 235}
\definecolor{pale2}{RGB}{175,238,238}
\definecolor{turquois4}{RGB}{0,134,139}

% Typesetting code
\definecolor{bg}{rgb}{0.95,0.95,0.95}
\usepackage{minted}
\usemintedstyle{friendly}
\newminted{python}{mathescape,frame=lines,framesep=4mm,bgcolor=bg}
\newminted{ipython}{mathescape,frame=lines,framesep=4mm,bgcolor=bg}
\newminted{julia}{mathescape,frame=lines,framesep=4mm,bgcolor=bg}
\newminted{c}{mathescape,linenos=true}
\newminted{r}{mathescape,  frame=none, baselinestretch=1, framesep=2mm}
\renewcommand{\theFancyVerbLine}{\sffamily
	\textcolor[rgb]{0.5,0.5,1.0}{\scriptsize {\arabic{FancyVerbLine}}}}


\usepackage{stmaryrd}

\newcommand{\Fact}{\textcolor{Brown}{\bf Fact. }}
\newcommand{\Facts}{\textcolor{Brown}{\bf Facts }}
\newcommand{\keya}{\textcolor{turquois4}{\bf Key Idea. }}
\newcommand{\Factnodot}{\textcolor{Brown}{\bf Fact }}
\newcommand{\Eg}{\textcolor{ForestGreen}{Example. }}
\newcommand{\Egs}{\textcolor{ForestGreen}{Examples. }}
\newcommand{\Ex}{{\bf Ex. }}
\newcommand{\Thm}{\textcolor{Brown}{\bf Theorem. }}
\newcommand{\Prf}{\textcolor{turquois4}{\bf Proof.}}
\newcommand{\Ass}{\textcolor{turquois4}{\bf Assumption.}} 
\newcommand{\Lem}{\textcolor{Brown}{\bf Lemma. }}

%source code 



% cali
\usepackage{mathrsfs}
\usepackage{bbm}
\usepackage{subfigure}

\newcommand{\argmax}{\operatornamewithlimits{argmax}}
\newcommand{\argmin}{\operatornamewithlimits{argmin}}

\newcommand\T{{\mathpalette\raiseT\intercal}}
\newcommand\raiseT[2]{\raisebox{0.25ex}{$#1#2$}}

\DeclareMathOperator{\cl}{cl}
%\DeclareMathOperator{\argmax}{argmax}
\DeclareMathOperator{\interior}{int}
\DeclareMathOperator{\Prob}{Prob}
\DeclareMathOperator{\kernel}{ker}
\DeclareMathOperator{\diag}{diag}
\DeclareMathOperator{\sgn}{sgn}
\DeclareMathOperator{\determinant}{det}
\DeclareMathOperator{\trace}{trace}
\DeclareMathOperator{\Span}{span}
\DeclareMathOperator{\rank}{rank}
\DeclareMathOperator{\cov}{cov}
\DeclareMathOperator{\corr}{corr}
\DeclareMathOperator{\range}{rng}
\DeclareMathOperator{\var}{var}
\DeclareMathOperator{\mse}{mse}
\DeclareMathOperator{\se}{se}
\DeclareMathOperator{\row}{row}
\DeclareMathOperator{\col}{col}
\DeclareMathOperator{\dimension}{dim}
\DeclareMathOperator{\fracpart}{frac}
\DeclareMathOperator{\proj}{proj}
\DeclareMathOperator{\colspace}{colspace}

\providecommand{\inner}[1]{\left\langle{#1}\right\rangle}

% mics short cuts and symbols
% mics short cuts and symbols
\newcommand{\st}{\ensuremath{\ \mathrm{s.t.}\ }}
\newcommand{\setntn}[2]{ \{ #1 : #2 \} }
\newcommand{\cf}[1]{ \lstinline|#1| }
\newcommand{\otms}[1]{ \leftidx{^\circ}{#1}}

\newcommand{\fore}{\therefore \quad}
\newcommand{\tod}{\stackrel { d } {\to} }
\newcommand{\tow}{\stackrel { w } {\to} }
\newcommand{\toprob}{\stackrel { p } {\to} }
\newcommand{\toms}{\stackrel { ms } {\to} }
\newcommand{\eqdist}{\stackrel {\textrm{ \scriptsize{d} }} {=} }
\newcommand{\iidsim}{\stackrel {\textrm{ {\sc iid }}} {\sim} }
\newcommand{\1}{\mathbbm 1}
\newcommand{\dee}{\,{\rm d}}
\newcommand{\given}{\, | \,}
\newcommand{\la}{\langle}
\newcommand{\ra}{\rangle}

\renewcommand{\rho}{\varrho}

\newcommand{\htau}{ \hat \tau }
\newcommand{\hgamma}{ \hat \gamma }

\newcommand{\boldx}{ {\mathbf x} }
\newcommand{\boldu}{ {\mathbf u} }
\newcommand{\boldv}{ {\mathbf v} }
\newcommand{\boldw}{ {\mathbf w} }
\newcommand{\boldy}{ {\mathbf y} }
\newcommand{\boldb}{ {\mathbf b} }
\newcommand{\bolda}{ {\mathbf a} }
\newcommand{\boldc}{ {\mathbf c} }
\newcommand{\boldi}{ {\mathbf i} }
\newcommand{\bolde}{ {\mathbf e} }
\newcommand{\boldp}{ {\mathbf p} }
\newcommand{\boldq}{ {\mathbf q} }
\newcommand{\bolds}{ {\mathbf s} }
\newcommand{\boldt}{ {\mathbf t} }
\newcommand{\boldz}{ {\mathbf z} }

\newcommand{\boldzero}{ {\mathbf 0} }
\newcommand{\boldone}{ {\mathbf 1} }

\newcommand{\boldalpha}{ {\boldsymbol \alpha} }
\newcommand{\boldbeta}{ {\boldsymbol \beta} }
\newcommand{\boldgamma}{ {\boldsymbol \gamma} }
\newcommand{\boldtheta}{ {\boldsymbol \theta} }
\newcommand{\boldxi}{ {\boldsymbol \xi} }
\newcommand{\boldtau}{ {\boldsymbol \tau} }
\newcommand{\boldepsilon}{ {\boldsymbol \epsilon} }
\newcommand{\boldmu}{ {\boldsymbol \mu} }
\newcommand{\boldSigma}{ {\boldsymbol \Sigma} }
\newcommand{\boldOmega}{ {\boldsymbol \Omega} }
\newcommand{\boldPhi}{ {\boldsymbol \Phi} }
\newcommand{\boldLambda}{ {\boldsymbol \Lambda} }
\newcommand{\boldphi}{ {\boldsymbol \phi} }

\newcommand{\Sigmax}{ {\boldsymbol \Sigma_{\boldx}}}
\newcommand{\Sigmau}{ {\boldsymbol \Sigma_{\boldu}}}
\newcommand{\Sigmaxinv}{ {\boldsymbol \Sigma_{\boldx}^{-1}}}
\newcommand{\Sigmav}{ {\boldsymbol \Sigma_{\boldv \boldv}}}

\newcommand{\hboldx}{ \hat {\mathbf x} }
\newcommand{\hboldy}{ \hat {\mathbf y} }
\newcommand{\hboldb}{ \hat {\mathbf b} }
\newcommand{\hboldu}{ \hat {\mathbf u} }
\newcommand{\hboldtheta}{ \hat {\boldsymbol \theta} }
\newcommand{\hboldtau}{ \hat {\boldsymbol \tau} }
\newcommand{\hboldmu}{ \hat {\boldsymbol \mu} }
\newcommand{\hboldbeta}{ \hat {\boldsymbol \beta} }
\newcommand{\hboldgamma}{ \hat {\boldsymbol \gamma} }
\newcommand{\hboldSigma}{ \hat {\boldsymbol \Sigma} }

\newcommand{\boldA}{\mathbf A}
\newcommand{\boldB}{\mathbf B}
\newcommand{\boldC}{\mathbf C}
\newcommand{\boldD}{\mathbf D}
\newcommand{\boldI}{\mathbf I}
\newcommand{\boldL}{\mathbf L}
\newcommand{\boldM}{\mathbf M}
\newcommand{\boldP}{\mathbf P}
\newcommand{\boldQ}{\mathbf Q}
\newcommand{\boldR}{\mathbf R}
\newcommand{\boldX}{\mathbf X}
\newcommand{\boldU}{\mathbf U}
\newcommand{\boldV}{\mathbf V}
\newcommand{\boldW}{\mathbf W}
\newcommand{\boldY}{\mathbf Y}
\newcommand{\boldZ}{\mathbf Z}

\newcommand{\bSigmaX}{ {\boldsymbol \Sigma_{\hboldbeta}} }
\newcommand{\hbSigmaX}{ \mathbf{\hat \Sigma_{\hboldbeta}} }

\newcommand{\RR}{\mathbbm R}
\newcommand{\CC}{\mathbbm C}
\newcommand{\NN}{\mathbbm N}
\newcommand{\PP}{\mathbbm P}
\newcommand{\EE}{\mathbbm E \nobreak\hspace{.1em}}
\newcommand{\EEP}{\mathbbm E_P \nobreak\hspace{.1em}}
\newcommand{\ZZ}{\mathbbm Z}
\newcommand{\QQ}{\mathbbm Q}


\newcommand{\XX}{\mathcal X}

\newcommand{\aA}{\mathcal A}
\newcommand{\fF}{\mathscr F}
\newcommand{\bB}{\mathscr B}
\newcommand{\iI}{\mathscr I}
\newcommand{\rR}{\mathscr R}
\newcommand{\dD}{\mathcal D}
\newcommand{\lL}{\mathcal L}
\newcommand{\llL}{\mathcal{H}_{\ell}}
\newcommand{\gG}{\mathcal G}
\newcommand{\hH}{\mathcal H}
\newcommand{\nN}{\textrm{\sc n}}
\newcommand{\lN}{\textrm{\sc ln}}
\newcommand{\pP}{\mathscr P}
\newcommand{\qQ}{\mathscr Q}
\newcommand{\xX}{\mathcal X}

\newcommand{\ddD}{\mathscr D}


\newcommand{\R}{{\texttt R}}
\newcommand{\risk}{\mathcal R}
\newcommand{\Remp}{R_{{\rm emp}}}

\newcommand*\diff{\mathop{}\!\mathrm{d}}
\newcommand{\ess}{ \textrm{{\sc ess}} }
\newcommand{\tss}{ \textrm{{\sc tss}} }
\newcommand{\rss}{ \textrm{{\sc rss}} }
\newcommand{\rssr}{ \textrm{{\sc rssr}} }
\newcommand{\ussr}{ \textrm{{\sc ussr}} }
\newcommand{\zdata}{\mathbf{z}_{\mathcal D}}
\newcommand{\Pdata}{P_{\mathcal D}}
\newcommand{\Pdatatheta}{P^{\mathcal D}_{\theta}}
\newcommand{\Zdata}{Z_{\mathcal D}}


\newcommand{\e}[1]{\mathbbm{E}[{#1}]}
\newcommand{\p}[1]{\mathbbm{P}({#1})}

%\theoremstyle{plain}
%\newtheorem{axiom}{Axiom}[section]
%\newtheorem{theorem}{Theorem}[section]
%\newtheorem{corollary}{Corollary}[section]
%\newtheorem{lemma}{Lemma}[section]
%\newtheorem{proposition}{Proposition}[section]
%
%\theoremstyle{definition}
%\newtheorem{definition}{Definition}[section]
%\newtheorem{example}{Example}[section]
%\newtheorem{remark}{Remark}[section]
%\newtheorem{notation}{Notation}[section]
%\newtheorem{assumption}{Assumption}[section]
%\newtheorem{condition}{Condition}[section]
%\newtheorem{exercise}{Ex.}[section]
%\newtheorem{fact}{Fact}[section]


\usepackage[T1]{fontenc}
\newtheorem{theorem}{Theorem}
\newtheorem{acknowledgement}{Acknowledgement}
\newtheorem{assumption}{Assumption}
\newtheorem{corollary}{Corollary}
\newtheorem{criterion}{Criterion}
\newtheorem{definition}{Definition}
\newtheorem{example}{Example}
\newtheorem{lemma}{Lemma}
\newtheorem{proposition}{Proposition}
\newtheorem{remark}{Remark}
\newtheorem{hypothesis}{Hypothesis}
\newtheorem{observation}{Observation}
\newenvironment{proof}[1][Proof]{\noindent\textbf{#1.} }{\ \rule{0.5em}{0.5em}}
%\input{tcilatex}

\makeatletter
\def\title@font{\Large\bfseries}
\let\ltx@maketitle\@maketitle
\def\@maketitle{\bgroup%
	\let\ltx@title\@title%
	\def\@title{\resizebox{\textwidth}{!}{%
			\mbox{\title@font\ltx@title}%
	}}%
	\ltx@maketitle%
	\egroup}
\makeatother
\usepackage{setspace}
\usepackage{amsmath}
\onehalfspacing
\newenvironment{p_enumerate}{
	\begin{enumerate}
		\setlength{\itemsep}{1pt}
		\setlength{\parskip}{0pt}
		\setlength{\parsep}{0pt}
	}{\end{enumerate}}
\sectionfont{\centering\mdseries\scshape\bfseries}
\subsectionfont{\raggedright\mdseries\scshape\bfseries}
\subsubsectionfont{\flushleft\mdseries\itshape\bfseries}
\makeatletter
\def\@seccntformat#1{\csname the#1\endcsname.\quad}
\makeatother
\def\signed #1{{\leavevmode\unskip\nobreak\hfil\penalty50\hskip2em
		\hbox{}\nobreak\hfil(#1)%
		\parfillskip=0pt \finalhyphendemerits=0 \endgraf}}
\newsavebox\mybox
\newenvironment{aquote}[1]
{\savebox\mybox{#1}\begin{quote}}
	{\signed{\usebox\mybox}\end{quote}}

\pdfminorversion=4

\makeatletter
\newcommand{\changeoperator}[1]{%
	\csletcs{#1@saved}{#1@}%
	\csdef{#1@}{\changed@operator{#1}}%
}
\newcommand{\changed@operator}[1]{%
	\mathop{%
		\mathchoice{\textstyle\csuse{#1@saved}}
		{\csuse{#1@saved}}
		{\csuse{#1@saved}}
		{\csuse{#1@saved}}%
	}%
}
\makeatother

\changeoperator{sum}
\changeoperator{
	prod}

\begin{document}
	
	
	
	
	

\vspace{-0.5ex}

%\begin{tabular}{l@{\hskip 4in}c@{\hskip 1in}c}
%	\textbf{June 2022} & \textbf{No. 2022: 6}
%	\vspace{0.15cm} \\
%\end{tabular}






\newpage{}



	
	\title{{Inflation Prediction Using Scanner Data  %\thanks {}
			%\title{{Money Demand Breakdown: An international Investigation
			}}
		
	
			
			%\date{This version: February, 2016\\
				%{First version: December, 2015}}
			\date{This version: July 2022}%\\This version: February, 2017}
		
		
		\author{Sonan Memon\footnote{Research Economist, PIDE, Islamabad. \texttt{smemon@pide.org.pk}}}
		
		
		
		\newpage{}
		
		\maketitle
		\vspace{-2ex}
		
		

		

	
	
	
		
		\begin{center}
			\line(1,0){470}
		\end{center}
		\begin{spacing}{1.1}
			\vspace{-3ex}
			\begin{abstract}
				\noindent 
				Predicting inflation using scanner data from super markets.
			\end{abstract}
		\end{spacing}
		\textbf{Keywords:} Inflation Prediction. Scanner Data From Retail Markets. Crisis of Hyperinflation. Russian and Ukraine Crisis. Machine Learning. {}\\
		\textbf{JEL Classification:}
		%\textbf{JEL Classification:}
		\\
		\begin{center}
			\vspace{-8ex}
			\line(1,0){470}
		\end{center}
		\pagenumbering{arabic}
		\baselineskip=18pt 
		
		\newpage{}
		
		\begin{figure}[H]
			\begin{center}
				\includegraphics[width=0.4\linewidth]{pidelogo.jpg}		
				\caption*{}
			\end{center}
		\end{figure}
	
	\vspace{-8ex}
		
		%	\begin{figure}[H]
		%	\begin{center}
			%	\includegraphics[width=0.5\linewidth]{crypto-pic.jpg}		
		%		\caption*{}
		%	\end{center}
	%	\end{figure}
	
	
	\tableofcontents
	
	\newpage{}
		
		
		\section{Introduction}
		
		Accurate forecasting of inflation is a concern for both market players and central banks. On the one hand, market participants want to update their inflation expectations in line with new information revelation so that their investment strategies are optimal. Meanwhile, central banks typically have mandates for price stability and they routinely collect data on inflation expectations and forecasts.
		
		
		While central banks collect data on consumer price indices, the frequency of such data does not allow accounting for sudden swings in expectations and their dramatic effects on inflation. Some examples of standard measures include the HICP (Harmonized Consumer Price Index) data used in the Euro area and the CPI (consumer price index) data from USA. Such data typically tends to be quarterly in worst cases or in best cases monthly, but results are revealed in the next month. However, when for instance, in a matter of few days and weeks, news about the Ukraine and Russian crisis changed the inflation expectations of many products, conventional price indices had little forecasting potential. The Ukraine and Russian crisis led to unexpected, sudden hyperinflation shock in many countries and similar inflation shocks can emerge from sudden change of State Bank governors or governments, terrorism episodes or political turmoil, especially in developing economies, where inflation tends to more volatile (see for instance \cite{vuletin2011replacing}).
		
		
		The State Bank of Pakistan (SBP) has also done some work on inflation forecasting by using neural network type machine learning methods and monthly year on year (YoY) inflation rate of Pakistan from Jan 1958 to Dec 2017 \cite{hanif2018thick}. Similarly, the SBP has worked on \textit{nowcasting} GDP using large scale manufacturing growth (LSM) in Pakistan \cite{hussain2018nowcasting} and LASSO type\footnote{Least Absolute Shrinkage Operator, Ridge Regressions and Elastic Nets.} ML methods. However, lack of high frequency data on the order of days or weeks poses a limitation in forecasting inflation using these methods. Hence, I argue that we need more granular data, on the order of days or weeks for enhancing forecasting. Next, I discuss methods for collecting such high frequency data, used at the current frontier of research on inflation.
		
		
		https://www.centralbanking.com/central-banks/economics/data/3315546/big-data-in-central-banks-2017-survey
		
		
		
		
		
		
		
		
		In recent literature, the daily consumer price index (CPI) produced by the Billion Prices Project (BPP CPI) of \cite{cavallo2016billion} offers a glimpse of the direction taken by consumer price inflation in \textit{real time}. For instance, Figure 1 is based on web scrapping online inflation data for Argentina \cite{cavallo2016billion}. It shows that the official CPI significantly under-stated actual inflation, when measured by web scrapping. An added benefit of such data is that it reveals the partisan measurement and particularly disclosure of CPI data in developing economies such as Argentina, where central bank independence is low. Should we expect a similar lack of correspondence between official inflation data of the SBP (State Bank of Pakistan) and non-partisan research measures?
		
		
		
		
		
		
		
		
		
		
		
		
		
		
		
		
		
		
				\begin{figure}[H]
				\begin{Center}
						\includegraphics[width=0.8\linewidth]{onlineversusCPI.png}
						\caption{Inflation in Argentina (Source is \cite{cavallo2016billion})}
					\end{Center}
				\end{figure}
		
		
		
		Meanwhile, another branch of emerging literature uses scanner-based data (see for instance \cite{beck2020price}) on prices rather than web scrapping. In Figure 2 below, recent scanner-based price indices for Germany are disclosed from the work of \cite{beck2022}. The data compares trajectories in 2022 (red and orange solid lines below) with their historical averages from 2019 to 2021 (blue and purple solid lines) along with historic minimum and maximum values (shaded areas). The data indicates a very strong increase in prices for sunflower oil and flour in light of the Ukraine conflict, accompanied by temporarily higher sales. The price increase of sunflower oil was rather gradual and already started as of early February. In contrast, prices for flour increased very sharply, but only more than two months after the invasion. However, in both cases, sales went far beyond their average levels, suggesting increased demand and possibly stockpiling behavior from pessimistic consumers (see \cite{cavallo2021can}). Concerning the more recent period up to June 2022, prices for both products seemed to have stabilized at a very high level, whereas quantities have converged back to their average levels.
		
		
		\begin{figure}[H]
			\begin{Center}
				\includegraphics[width=0.8\linewidth]{germany_inflationforecastscanner.png}
				\caption{Source is \cite{beck2022}}
			\end{Center}
		\end{figure}
		
		
		
		
		
		
	 
		
		
		
		
		
		
		\section{More on Scanner Data}
		
		
		\cite{modugno2013now}.
		
		\cite{beck2020price}.
		
		\newpage{}
		
		\section{ML and Prediction}
		
		\newpage{}
		
		\section{Conclusion}
		
		
		
		
		
		
		%	\begin{figure}[H]
		%	\begin{Center}
		%		\includegraphics[width=0.8\linewidth]{bitcoin_electricity.png}
		%		\caption{Source is \textit{Cambridge Center for Alternative Finance}}
		%	\end{Center}
	%	\end{figure}
		
		
		
		
		
		
	
		
		
		
		
		
			\newpage
		%_________________ End of Main Matter_________________%
		%_________________ Reference Section _______________%
		\phantomsection % allows for correct link to Table of Contents
		\addcontentsline{toc}{section}{References} % Adds the line "References" to Table of contents
		\singlespacing
		%\bibliography{references} % Uses the Bibtex-file mybibfile.bib
		\newpage
		\bibliographystyle{aer}
		\bibliography{references}
		\clearpage
		%_________________ Space for Supplementary Material _______________%
		
		
		
		
		%\section*{Appendix}
		
		
		
		
		
	\end{document}


